\apendice{Plan de Proyecto Software}

En este apéndice se describirá la planificación que se ha seguido a la hora de organizar el proyecto, tanto desde la perspectiva temporal y cómo se ha distribuido el trabajo como desde un estudio de la viabilidad del mismo.

\section{Planificación temporal}

La organización temporal de este trabajo se ha realizado mediante Sprints de una duración variable entre 1 y 2 semanas, en función de la carga de trabajo de cada uno de ellos.

\subsection{\textit{Sprint} 1}
Este se dedico a la lectura de los artículos científicos y la interiorización de los contenidos expuestos.

\subsection{\textit{Sprint} 2}
En este se Inicio un proceso de familiarización con los datos, tratando de visualizarlos y comprender sus estructuras internas.

\subsection{\textit{Sprint} 3}
En este Sprint se inicio el Notebook y se aplicaron los conocimientos de la fase anterior, componiendo las imágenes y capas.

\subsection{\textit{Sprint} 4}
Este sprint fue dedicado a la creación y entrenamiento de los modelos sin contexto de los pixeles adyacente, es decir KNN, SVM y Random Forest. Además de medir sus métricas.

\subsection{\textit{Sprint} 5}
Este sprint fue dedicado al desarrollo del modelo de la red U-Net.

\subsection{\textit{Sprint} 6}
Este fue dedicado al entrenamiento y visualización de las primeras predicciones.

\subsection{\textit{Sprint} 7}
Este Sprint finalmente fue dedicado a la evaluzación de los modelos de la red U-Net.

\section{Estudio de viabilidad}

En este apartado se va a comentar la viabilidad del proyecto tanto de forma económica, calculando el coste total que debería de haber costado el desarrollo del proyecto, y la viabilidad legal de las librerías y herramientas utilizadas.

\subsection{Viabilidad económica}

En este subapartado se encuentran los cálculos económicos del proyecto general.
Estos son los derivados del equipo y personal necesarios para la realización del mismo ya que el resto de los materiales son de libre acceso.

Para realizar el cálculo se han divido los gastos en:
\begin{itemize}
	\item \textbf{Coste personal} (tabla~\ref{tab:costes_personal}): contratación del personal de investigación y desarrollo del proyecto.
	\item \textbf{Coste \textit{hardware}} (tabla~\ref{tab:costes_hardware}): dispositivos \textit{hardware} necesarios en el proyecto.
\end{itemize}

\begin{table}\centering
	\begin{tabular}[]{@{}l r@{}}
		\toprule
		\textbf{Concepto} & \textbf{Coste (\euro{})} \\
		\midrule
		Salario mensual bruto & 2.047,78 \\
		Seguridad Social (30,04\%) & 615,15 \\
		Retención IRPF (2\%) & 28,65 \\
		Salario mensual neto & 1.403,97 \\\hubu
		\textbf{Total 3 meses } &  6.143,34 \\
		\bottomrule
	\end{tabular}
	\caption{Costes de personal.}
	\label{tab:costes_personal}
\end{table}

\begin{table}
	\centering
	\begin{tabular}[]{@{}l c r@{}}
		\toprule
		\textbf{Concepto} & \textbf{Coste (\euro{})} \\
		\otoprule
		Ordenador de desarrollo (x1) & 950 \\\hubu
		\textbf{Total} & 950 \\
		\bottomrule
	\end{tabular}
	\caption{Costes de \textit{hardware}.}
	\label{tab:costes_hardware}
\end{table}

Finalmente se ha realizado un cálculo final del coste del proyecto, tabla~\ref{tab:coste_final}.

\begin{table}
	\centering
	\begin{tabular}[]{@{}l r@{}}
		\toprule
		\textbf{Tipo} & \textbf{Coste (\euro{})}\\
		\otoprule
		Personal  & 6.143,34 \\
		\textit{Hardware}& 950 \\\hubu
		\textbf{Total}&7.093,34\\
		\bottomrule
	\end{tabular}
	\caption{Costes final.}
	\label{tab:coste_final}
\end{table}

\subsection{Viabilidad legal}
En este subapartado se van a comentar las distintas licencias que tienen las herramientas y librerías utilizadas en el proyecto, así como se comenta la licencia final que tiene el proyecto.

Las licencias de las librerías y herramientas utilizadas en el desarrollo del proyecto se pueden ver en la tabla~\ref{tab:lic}.

\begin{table}[h]
	\centering
	\begin{tabular}{lc}
		\toprule
		\textbf{Librería-Herramienta}&\textbf{Licencia}\\
		\midrule
		\textit{Numpy} & BSD 3\\
		\textit{Matplotlib} &  PSF\\
		\textit{Seaborn} & BSD 3\\
		\textit{Python} & PSF\\
		\textit{Scikit-learn} & BSD 3\\
		\textit{Tensorflow} & Apache 2.0\\
		\bottomrule
	\end{tabular}
	\caption{Tabla con las licencias de las librerías y herramientas utilizadas.}
	\label{tab:lic}
\end{table}

Teniendo en cuenta estas licencias, las licencias de las herramientas usadas tanto en la aplicación final de rehabilitación y las licencias de las herramientas y librerías usadas por el compañero se ha decido utilizar la licencia \textit{GPL v3.0} a partir de las licencias más restrictivas.

Con esta licencia \textit{GPL v3.0} se puede utilizar el \textit{software} desarrollado para uso comercial, se puede modificar las implementaciones realizadas,  distribuirlas, realizar patentes sobre ellas y usarlas de forma privada.