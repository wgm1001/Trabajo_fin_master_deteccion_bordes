\chapter{Objetivos}

En esta sección se detallarán los objetivos a cumplir en este trabajo.

\section*{Objetivo Principal}
El objetivo principal de este trabajo de fin de máster es desarrollar un modelo de detección de bordes de parcelas eficiente y preciso que pueda ser aplicado a imágenes satélite de alta resolución. Se buscará entrenar un modelo de inteligencia artificial capaz de diferenciar de manera eficiente y acertada los bordes de las parcelas en este tipo de imágenes.

\section*{Objetivos específicos}
Para lograr el objetivo principal, se plantean los siguientes objetivos específicos:

\begin{itemize}
	\item Entrenar un modelo capaz de realizar una separación suficiente de parcelas en base a una imagen satelital. 
	\item Probar diferentes métodos y configuraciones de cara a comprobar si realmente la red neuronal convolucional es el método más potente o pueden ser más interesantes otras metodologías que impliquen que no tener en cuenta el contexto de los píxeles.
	\item Exponer los conocimientos extraídos en un notebook de Python, lo cuál facilitará la trasmisión y la manipulación de estos.
	\item Generar un repositorio dónde almacenar dicho notebook con la estructura definida para una poder ejecutar de forma simple el cuaderno.
	\item Tratar de optimizar en la medida de lo posible los parámetros del modelo para maximizar su eficiencia.
\end{itemize}